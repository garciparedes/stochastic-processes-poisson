% !TEX root = ./document.tex

\documentclass[a4paper, spanish]{article}

\usepackage[utf8]{inputenc}
\usepackage[spanish, es-tabla]{babel}
\usepackage[margin={20mm, 22mm}]{geometry}
\usepackage[T1]{fontenc}
\usepackage{dirtytalk}


\usepackage{amsmath}
\usepackage{amssymb}

\usepackage{titlesec}
\renewcommand\thesection{\alph{section})}
\titleformat{\section}
{\normalfont\bfseries}{\thesection}{1.0em}{}

\usepackage{csquotes}

\title{Procesos Estocásticos: Ejercicios de Poisson}
\author{
  Gabriel Rodríguez Canal \and
  Marcos Ventura Conde Osorio \and
  Sergio García Prado
}
\date{\today}

\begin{document}

  \maketitle

  \section{Ejercicio 2.37 - Durret}

    \paragraph{}
    El enunciado del ejercicio se indica a continuación:

    \begin{displayquote}
      \say{As a community service members of the Mu Alpha Theta fraternity are going to pick up cans from along a roadway. A Poisson mean $60$ members show up for work. Two-third of the workers are enthusiastic and will pick up a mean of ten cans with a standard deviation of $5$. One-third of the workers are lazy and will only pick up an average of three cans with a standard deviation of $2$. Find the mean and standard deviation of the the number of cans collected.}
    \end{displayquote}

    \paragraph{}
    Lo primero es indicar la notación que utilizaremos para plantear el problema:

    \begin{align*}
      M &\equiv \text{Número de miembros que van a recoger latas en cada excursión.} \\
      p &\equiv \text{Proporción de miembros entusiastas respecto del total.} \\
      M_E &\equiv \text{Número de miembros entusiastas que van a recoger latas.} \\
      M_L &\equiv \text{Número de miembros perezosos que van a recoger latas.} \\
      E_i &\equiv \text{Número de latas recogidas por el miembro entusiastas i-ésimo} \\
      L_j &\equiv \text{Número de latas recogidas por el miembro perezoso j-ésimo} \\
      S &\equiv \text{Número de latas recogidas en total.} \\
      S_E &\equiv \text{Número de latas recogidas por miembros entusiastas en total.} \\
      S_L &\equiv \text{Número de latas recogidas por miembros perezos en total.}
    \end{align*}

    \paragraph{}
    Además, el enunciado indica las siguientes propiedades acerca de estas variables aleatorias:

    \begin{align*}
      \lambda &= 60 \\
      p &= \frac{2}{3} \\
      M &\sim Poisson(\lambda) \equiv Poisson(60) \\
      E[E_i] &= 10 \\
      Var[E_i] &= 5 ^ 2 = 25 \\
      E[L_j] &= 3 \\
      Var[L_j] &= 2^2 = 4
    \end{align*}

    \paragraph{}
    Aplicando la propiedad de \emph{thinning} sobre el número de miembros de cada tipo tenemos que:

    \begin{align*}
      M_E &\sim Poisson(p \cdot \lambda) \equiv Poisson(40) \\
      M_L &\sim Poisson((1 - p) \cdot \lambda) \equiv Poisson(20)
    \end{align*}

    \paragraph{}
    Además $M_E$ y $M_L$ son independientes entre si. Por tanto, podemos descomponer la suma de latas recogidas en dos subgrupos que podemos calcular por separado:

    \begin{align*}
      S = S_E + S_L
    \end{align*}

    \paragraph{}
    Para el cálculo de esperanzas aplicamos la propiedad de la suma de $N$ variables $X_i$ \emph{i.i.d}, que indica que $E[X_1 + ... + X_N] = E[X_i] \cdot E[N]$. A continuación se muestran los cálculos pertinentes:

    \begin{align*}
      E[S_E]
      &= E[E_i] \cdot E[M_E]\\
      &= 10 * 40 \\
      &= 400 \\
      \\
      E[S_L]
      &= E[L_j] \cdot E[M_L] \\
      &= 3 * 20 \\
      &= 60 \\ 
      \\
      E[S]
      &= E[S_E] + E[S_L] \\
      &= 400 + 60 \\
      &= 460
    \end{align*}

    \paragraph{}
    Para el cálculo de varianzas aplicamos la propiedad de la suma de $N$ variables $X_i$ \emph{i.i.d}, que indica que $Var[X_1 + ... + X_N] = E[N] \cdot Var[X_i] + Var[N] \cdot E[X_i]^2$. A continuación se muestran los cálculos pertinentes:

    \begin{align*}
      Var[S_E]
      &= E[M_E] \cdot Var[E_i] + Var[M_E] \cdot (E[E_i])^2 \\
      &= 40 \cdot 25 + 40 \cdot 10 ^ 2 \\
      &= 5000\\
      \\
      Var[S_L]
      &= E[M_L] \cdot Var[L_j] + Var[M_L] \cdot (E[L_j])^2 \\
      &= 20  \cdot 4 + 20 \cdot 3 ^ 2\\
      &= 260 \\ 
      \\
      Var[S]
      &= Var[S_E] + Var[S_L] \\
      &= 5000 + 260 \\
      &= 5260
      \\
      SD[S]
      &= \sqrt{Var[S]} \\
      &= \sqrt{5260} \\
      &\approx 72.5
    \end{align*}

    \paragraph{}
    Por tanto, la solución al ejercicio es la siguiente: \textbf{En promedio, se recogen $460$ latas, con una desviación estándar de $72.5$ latas aproximadamente}.

  \section{Ejercicio 2.52 - Durret}

  \paragraph{}
  El enunciado del ejercicio se indica a continuación:

  \begin{displayquote}
    \say{A light bulb has a life time that is exponential with a mean of 200 days. When it burns out a janitor replaces it immediately. In addition there is a handyman who comes at times of a Poisson process at rate 0.01 and replaces the bulb as \say{preventive maintenance.} (a) How often is the bulb replaced? (b) In the long run what fraction of the replacements are due to failure?}
  \end{displayquote}

  \paragraph{}
  Lo primero es indicar la notación que utilizaremos para plantear el problema:

  \begin{align*}
    T_C(n) &\equiv \text{Tiempo en días hasta que un conserje cambie la bombilla por n-ésima vez.} \\
    T_M(n) &\equiv \text{Tiempo en días hasta que un encargado de mantenimiento cambie la bombilla por n-ésima vez.}
  \end{align*}

  \paragraph{}
  Además, el enunciado indica las siguientes propiedades acerca de estas variables aleatorias:

  \begin{align*}
    \lambda_C(n) &= \frac{1}{200 \cdot n} \\
    \lambda_M(n) &= \frac{1}{100 \cdot n} \\
    T_C(n) &\sim Exp(\lambda_C(n)) \equiv Exp\left(\frac{1}{200 \cdot n}\right) \\
    T_M(n) &\sim Exp(\lambda_M(n)) \equiv Exp\left(\frac{1}{100 \cdot n}\right)
  \end{align*}

  \paragraph{}
  Aplicando la propiedad de \emph{superposition} entre los dos procesos (que asumimos independientes) de cambio de bombilla definimos:

  \begin{align*}
    T(n) \sim Exp(\lambda_C(n) + \lambda_M(n))
    &\equiv Exp\left(\frac{1}{100 \cdot n} +\frac{1}{200 \cdot n} \right) \\
    &\equiv Exp\left(\frac{3}{200 \cdot n}\right)
  \end{align*}

  \paragraph{}
  Entonces, el tiempo esperado para que se cambie una bombilla es:

  \begin{align*}
    E[T(1)]
    &= \frac{1}{\lambda_C(1) + \lambda_M(1))} \\
    &= \frac{200}{3} \\
    &\approx 66. 6
  \end{align*}

  Por tanto, la solución al apartado (a) es la siguiente: \textbf{La bombilla se cambia en promedio cada 66.6 días aproximadamente}.

  \paragraph{}
  En cuanto al apartado (b), se pregunta pro la fracción a largo plazo referida a que una bombilla se funda antes de cambiarla en buen estado. Esto se puede calcular como la probabilidad de que el suceso $T_C$ (cambio por rotura) ocurra antes que $T_M$ (cambio preventivo). Esto se puede calcular de la siguiente manera:

  \begin{align*}
    \lim_{n\to\infty} P(T_C(n) < T_M(n))
    &= \frac{\lambda_C(n)}{\lambda_C(n) + \lambda_M(n)} \\
    &= \frac{\frac{1}{200 \cdot n}}{\frac{3}{200 \cdot n}}\\
    &= \frac{1}{3}
  \end{align*}

    Por tanto, la solución al apartado (b) es la siguiente: \textbf{La probabilidad de que se funda la bombilla antes de que la cambien es de $\frac{1}{3}$}.


  \section{Ejercicio 2.60 - Durret}

    \paragraph{}
    El enunciado del ejercicio se indica a continuación:


    \begin{displayquote}
      \say{Suppose that the number of calls per hour to an answering service follows a Poisson process with rate 4. Suppose that 3/4’s of the calls are made by men, 1/4 by women, and the sex of the caller is independent of the time of the call. (a) What is the probability that in 1 h exactly two men and three women will call the answering service? (b) What is the probability three men will make phone calls before three women do?}
    \end{displayquote}


    \paragraph{}
    Lo primero es indicar la notación que utilizaremos para plantear el problema:

    \begin{align*}
      C &\equiv \text{Número de llamadas por hora.} \\
      p &\equiv \text{Proporción de hombres respecto del total.} \\
      C_H &\equiv \text{Número de miembros entusiastas que van a recoger latas.} \\
      C_M &\equiv \text{Número de miembros perezosos que van a recoger latas.}
    \end{align*}

    \paragraph{}
    Además, el enunciado indica las siguientes propiedades acerca de estas variables aleatorias:

    \begin{align*}
      \lambda &= 4 \\
      p &= \frac{3}{4} \\
      C &\sim Poisson(\lambda) \equiv Poisson(4)
    \end{align*}

    \paragraph{}
    Aplicando la propiedad de \emph{thinning} sobre el número de miembros de cada tipo tenemos que:

    \begin{align*}
      C_H &\sim Poisson(p \cdot \lambda) \equiv Poisson(3) \\
      C_M &\sim Poisson((1 - p) \cdot \lambda) \equiv Poisson(1)
    \end{align*}

    \paragraph{}
    Aplicando la propiedad de independencia $P(A, B) = P(A) \cdot P(B)$ calculamos la probabilidad conjunta de que llamen exactamente $2$ hombres y $3$ mujeres. Los cálculos se muestran a continuación:

    \begin{align*}
      P(C_H = 2, C_M = 3)
      &= P(C_H = 2) \cdot P(C_M = 3) \\
      &= \frac{e^{-3} \cdot 3^2}{2!} \cdot \frac{e^{-1} \cdot 1^3}{3!} \\
      &\approx 0.0137
    \end{align*}

    \paragraph{}
    Por tanto, la solución al apartado (a) del ejercicio es la siguiente: \textbf{La probabilidad de que en una hora llamen exactamente $2$ hombres y $3$ mujeres es de $0.0137$ approximadamente}.

    \paragraph{}
    Para el apartado (b), vamos a aplicar la propiedad que indica que cuando se fija el número de sucesos posibles en un proceso de Poisson, entonces el número de casos favorables se puede probabilizar a partir de la distribución Binomial. La justificación es la siguiente:

    \paragraph{}
    Si  estudiamos 2 procesos $N_1 \sim Poisson(\lambda_1)$, $N_2 \sim Poisson(\lambda_2)$ independientes, considerando $N_1$ como casos favorables y $N_2$ como casos desfavorables, al fijar el número de casos totales en $n$, entonces la distribucion $Y \sim Bin\left(n, \frac{\lambda_1}{\lambda_1 + \lambda_2}\right)$ permite estudiar el número de casos favorables obtenidos.

    \paragraph{}
    Por tanto, para nuestro caso particular definimos $X \sim Bin\left(5, \frac{3}{4}\right)$ dado que el suceso de que llamen $3$ hombres antes que $3$ mujeres se da estudiando $5$ llamadas consecutivas y la proporción de llamadas realizadas por hombres es de $\frac{3}{4}$.

    \begin{align*}
      P(X \geq 3)
      &= \sum_{i = 3}^6 \binom{6}{k} \cdot \left(\frac{3}{4} \right)^k \cdot \left(1 - \frac{3}{4}\right) ^ {6 - k} \\
      &\approx 0.633
    \end{align*}

    \paragraph{}
    Por tanto, la solución al apartado (b) del ejercicio es la siguiente: \textbf{La probabilidad de que lleguen $3$ llamadas realizadas por hombres antes que $3$ llamadas realizadas por mujeres es de $0.633$ approximadamente}.

\end{document}
